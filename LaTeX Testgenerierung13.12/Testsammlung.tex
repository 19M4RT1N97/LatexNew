% @definitions
\documentclass{exam}
%\usepackage[utf8]{inputenc}

\usepackage[utf8]{inputenc}
\usepackage[ngerman]{babel}

\usepackage{pgfplots} % for \includegraphics
\pgfplotsset{compat=1.14}

% all for source code
\usepackage{algorithm}
\usepackage{algorithmic}
\usepackage{listings}
\usepackage{color}
\definecolor{pblue}{rgb}{0.13,0.13,1}
\definecolor{pgreen}{rgb}{0,0.5,0}
\definecolor{pred}{rgb}{0.9,0,0}
\definecolor{pgrey}{rgb}{0.46,0.45,0.48}
\lstset{language=Java,
  showspaces=false,
  showtabs=false,
  breaklines=true,
  showstringspaces=false,
  breakatwhitespace=true,
  commentstyle=\color{pgreen},
  keywordstyle=\color{pblue},
  stringstyle=\color{pred},
  basicstyle=\ttfamily,
  moredelim=[il][\textcolor{pgrey}]{$$},
  moredelim=[is][\textcolor{pgrey}]{\%\%}{\%\%},
	numbers=left,
	numberstyle=\tiny,
  basicstyle=\ttfamily\footnotesize
}

% remove left checkbox margin
\renewcommand{\checkboxeshook}{
  \settowidth{\leftmargin}{0pt}
}

\begin{document}

% @header
\pagestyle{headandfoot}
\firstpageheader{POS1/2AHIF/A}{\huge{\textbf{Test Sommersemester 2016}}}{\today}

% @footer
\footer{}{
\begin{tabular}{l || c | c | c | c | c}
  \textbf{Note} & \hspace{12pt} sehr gut \hspace{12pt} & \hspace{22.5pt} gut \hspace{22.5pt} & \hspace{3pt} befriedigend \hspace{3pt} & \hspace{9pt} genügend \hspace{9pt} & nicht genügend \\
  \hline
  \textbf{Punkte} & 42 - 48 & 36 - 41 & 30 - 35 & 24 - 29 & 0 - 23
 \end{tabular}
}{\thepage}

% @result
\begin{flushleft}
  \Large{
    \begin{tabular}{l l l}
      Name \hspace{0.4\textwidth} & Punkte \hspace{0.15\textwidth} & Note
    \end{tabular}
  }
\end{flushleft}

% @description
\begin{center}
  \fbox{
    \parbox{0.95\textwidth}{
      \begin{center}
      \textbf{Testbeschreibung}
      \end{center}
      \textbf{Single-Choice-Fragen} \\
      Bei dieser Art von Fragen ist nur eine Antwort richtig. Das Symbol zur Kennzeichnung neben der Aufgabenstellung ist $(=1)$. Single-Choice-Fragen werden jeweils mit 4 Pluspunkten bewertet, wenn sie vollständig richtig beantwortet sind. Sind sie (teilweise) falsch beantwortet, gibt es 0 Punkte. \\

      \textbf{Multiple-Choice-Fragen} \\
      Bei dieser Art von Fragen ist mehr als eine Antwort richtig. Das Symbol zur Kennzeichnung neben der Aufgabenstellung ist $(>1)$. Bei Multiple-Choice-Fragen werden 4 Pluspunkte auf die richtigen Antworten aufgeteilt. Ebenso werden 4 Minuspunkte auf die falschen Antwortmöglichkeiten aufgeteilt. Insgesamt kann eine solche Frage mit maximal 4 Punkten und minimal mit 0 Punkten bewertet werden.
    }
  }
\end{center}

\begin{questions}
  
  % @name=ArrayList remove-Reihenfolge
  % @topic=Listen
  % @difficulty=2
  % @size=0.5
  \begin{minipage}{\linewidth}
    \question In welcher Reihenfolge sollten die folgenden Operationen der Methode \lstinline[columns=fixed]{remove(Object del)} bei einer \textit{ArrayList} durchgeführt werden? Trage die Zahlen $1$ (als erstes) bis $4$ (als letztes) in die orangefarbenen Kreise ein. $(=1)$ 
    \begin{center}
        \includegraphics[width=0.7\textwidth]{pictures/arrayList_removeAt.png}
    \end{center}
    \vspace{12pt}
  \end{minipage}
  
  % @name=Aufrufkette Reihenfolge
  % @topic=Rekursion
  % @difficulty=2
  % @size=0.333
  \begin{minipage}{\linewidth}
    \question Welche Aufrufkette erzeugt der Aufruf folgender Rekursion mit $n=2$? $(=1)$
    \begin{lstlisting}
  int sumRec(int n)
  {
    if (n == 0) {
      return 0;
    }
    else {
      return n + sumRec(n - 1);
    }
  }
    \end{lstlisting}
    \begin{checkboxes}
      \choice \lstinline[columns=fixed]{sumRec(2)} $\rightarrow$ \lstinline[columns=fixed]{return 2 + sumRec(1)} 
      \choice \lstinline[columns=fixed]{sumRec(2)} $\rightarrow$ \lstinline[columns=fixed]{2 + sumRec(1)} $\rightarrow$ \lstinline[columns=fixed]{1 + sumRec(0)} $\rightarrow$ \lstinline[columns=fixed]{return 0}
      \choice \lstinline[columns=fixed]{sumRec(2)} $\rightarrow$ \lstinline[columns=fixed]{sumRec(1)} $\rightarrow$ \lstinline[columns=fixed]{sumRec(0)} $\rightarrow$ \lstinline[columns=fixed]{return 0}
      \choice \lstinline[columns=fixed]{sumRec(2)} $\rightarrow$ \lstinline[columns=fixed]{sumRec(1)} $\rightarrow$ \lstinline[columns=fixed]{sumRec(0)} $\rightarrow$ \lstinline[columns=fixed]{return 0} $\rightarrow$ \lstinline[columns=fixed]{return 1 + 0} $\rightarrow$ \lstinline[columns=fixed]{return 2 + 1}
    \end{checkboxes}
    \vspace{12pt}
  \end{minipage}

  % @name=Aussagen zur Rekursion/Iteration
  % @topic=Rekursion
  % @difficulty=1
  % @size=0.25
  \begin{minipage}{\linewidth}
    \question Welche Aussagen sind wahr? $(>1)$
    \begin{checkboxes}
      \choice Jede primitive Rekursion kann durch eine einzelne Iteration ersetzt werden.
      \choice Jede Iteration ist eine Rekursion.
      \choice Rekursionen können nicht durch Iterationen ersetzt werden.
      \choice Jede einzelne Iteration kann durch eine primitive Rekursion ersetzt werden.
    \end{checkboxes}
    \vspace{12pt}
  \end{minipage}

\end{questions}

\end{document}
